\documentclass {article}
\usepackage[czech]{babel}
\usepackage[utf8]{inputenc}

\usepackage[backend=bibtex,style=ieee]{biblatex}
\bibliography{papr4} 

\usepackage{amsthm}
\usepackage{amsmath}

\theoremstyle {definition}
\newtheorem{example}{Příklad}
\newtheorem{exercise}{Cvičení}
\newtheorem*{remark}{Poznámka}

\usepackage{hyperref}

\begin {document}
\section {Monitor}
Monitor je objekt, který zabaluje data a procedury (tak, jak to známe
z OOP), ale navíc také synchronizaci. Metody monitoru jsou všechny
synchronizované, tedy pouze jedno vlákno může zaráz vykonávat
libovolnou metodu monitoru. Navíc monitor může obsahovat podmínky, ke
kterým se dostaneme.

\begin {example}
Jednoduchým příkladem je synchronizovaný čítač. V naší knihovně bychom
podědili třídu \texttt{MONITOR}.

\begin{verbatim}
(defclass counter (monitor)
  ((count :initform 0)))
\end{verbatim}

Inkrementaci pak definujeme makrem \texttt{DEFINE-MONITOR-METHOD}.

\begin{verbatim}
(define-monitor-method increment ((counter counter))
  (incf (slot-value counter 'count)))
\end{verbatim}

Je důležité, aby třída, která dědí z monitoru byla uvedena jako první
argument.
\end {example}

Monitor si můžeme představit tak, že obsahuje mutex, na který se čeká
na začátku každé metody a na konci každé metody se
signalizuje. Nicméně pokud je potřeba v rámci metody počkat než bude
splněna nějaká podmínka, můžeme monitoru přiřadit takzvanou podmíněnou
proměnnou, na kterou lze čekat, což uspí vlákno a signalizuje mutex,
aby ostatní vlákna měly možnost podmínku splnit. Splnění podmínky je
pak potřeba signalizovat.

\begin {remark}
  Je dobré podotknout, že mutex, bežně nazývaný také zámek, musí být
  takzvaně reentrantní, nebo-li aby jedno vlákno mohlo vícekrát
  signalizovat tento mutex. Pokud by reentrantní nebyl, nastal by
  deadlock, a to v případě, že v rámci jedné metody monitoru je zavolána jiná
  metoda monitoru.
\end {remark}

\begin {example}
  Pokud bychom chtěli, aby čítač šlo také dekrementovat, ale pouze
  pokud je hodnota kladná, přidáme podmíněnou proměnnou.

\begin{verbatim}
(defclass counter (monitor)
  ((count :initform 0)
   (positivep :initform (condition-variable))))
\end{verbatim}

Samotná dekrementace pak podmíněnou proměnnou využívá.

\begin{verbatim}
(define-monitor-method decrement ((counter counter))
  (with-slots (count positivep) counter
    (loop :until (plusp count)
       :do (wait-on-condition positivep))
    (decf count)))
\end{verbatim}

Inkrementace pak musí signalizovat, že hodnota už je znovu kladná.

\begin{verbatim}
(define-monitor-method increment ((counter counter))
  (incf (slot-value counter 'count))
  (signal-condition (slot-value counter 'positivep)))
\end{verbatim}
\end {example}

Podmíněné proměnné, které používáme jsou neblokující, což znamená, že
vlákno, které signalizuje proměnnou není blokováno a dokončí svou
práci. To má ale za následek, že je potřeba po každém čekání
zkontrolovat, jestli už je podmínka splněna.

\begin {remark}
  Existují ještě další typy podmíněných proměnných. Jednou z nich je
  blokující, která blokuje vlákno, které signalizuje proměnnou. Tento
  typ byl prvním návrhem \cite{hoare74,hansen72}. Druhou pak je
  implicitní, která nespecifikuje podmíněné proměnné, ale je obsažena
  jedna implicitní podmínka pro jeden monitor. Pak tedy není potřeba
  žádnou vytvářet, pouze se čeká na a signalizuje samotný
  monitor. Tento typ je například použit v Javě, kde metody monitoru
  jsou označeny klíčovým slovem \texttt {synchronized} a pro
  podmíněnou proměnnou používáme \texttt {wait} a
  \texttt{notify}/\texttt {notifyAll}.

  Neblokující podmíněné proměnné se používají např. v Pthread nebo
  také ve většině implementací CL. Lze je také najít v knihovně
  \texttt {java.util.concurrent.locks}.

  Schematické obrázky vysvětlující princip fungování jednotlivých typů
  lze najít na
  \href{https://en.wikipedia.org/wiki/Monitor_(synchronization)#Blocking_condition_variables}{wikipedii}.
\end {remark}

\subsection {Producent-konzument}
Řešením tohoto problému je v podstatě vytvoření sdílené blokující
fronty, která bude mít metody \texttt{enqueue} a
\texttt{dequeue}. Použijeme podobnou terminologii a řešení jako při
použití semaforů.

\begin{verbatim}
(defclass queue (monitor)
  ((itemsp :initform (make-instance 'condition-variable))
   (spacep :initform (make-instance 'condition-variable))
   (buffer :initform nil)
   (capacity :initform 1 :initarg :capacity)))
(defun queue (capacity)
  (make-instance 'queue :capacity capacity))
\end{verbatim}

Podmíněná proměnná \texttt {itemsp} značí konzumentům, že existuje
nějaký prvek v bufferu a naproti tomu \texttt {spacep} značí
producentům, že je místo v bufferu.

\begin{verbatim}
(define-monitor-method enqueue ((queue queue) value)
  (with-slots (buffer capacity spacep itemp) queue
    (loop :while (= (length buffer) capacity)
          :do (wait-on-condition spacep))
    (setf buffer `(,@buffer ,value))
    (signal-condition itemp))
  t)
                 
(define-monitor-method dequeue ((queue queue))
  (with-slots (buffer capacity spacep itemp) queue
    (loop :while (null buffer)
          :do (wait-on-condition itemp))
    (let ((value (pop buffer)))
      (signal-condition spacep)
      value)))
\end{verbatim}


\subsection {Čtenáři-písaři}
Abychom tento problém vyřešili dostatečně obecně, vytvoříme třídu,
která bude sloužit jako zámek do místnosti, kde budou dokumenty, které
se budou číst a do kterých se bude zapisovat. Bude mít dvoje dveře,
jedny pro čtenáře a jedny pro písaře, jako tomu bylo u řešení pomocí
semaforů. Budeme mít tedy čtyři metody monitoru \texttt{reader-enter},
\texttt{reader-leave}, \texttt{writer-enter} a \texttt{writer-enter}.

\begin{verbatim}
(defclass read-write-lock (monitor)
  ((emptyp :initform (make-instance 'condition-variable))
   (writerp :initform nil)
   (readers :initform 0)))
(defun read-write-lock ()
  (make-instance 'read-write-lock))
\end{verbatim}

Použijeme pouze jednu podmíněnou proměnnou \texttt {emptyp}, která
značí, že místnost je prázdná. Dále proměnná \texttt{writerp}, značí,
že v místnosti je čtenář a proměnná \texttt{readers} bude uvádět počet
čtenářů v místnosti.

\begin{verbatim}
(define-monitor-method reader-enter ((rw read-write-lock))
  (with-slots (writerp readers emptyp) rw
    (loop :while writerp
          :do (wait-on-condition emptyp))
    (incf readers))
  t)

(define-monitor-method reader-leave ((rw read-write-lock))
  (with-slots (writerp readers emptyp) rw
    (decf readers)
    (when (zerop readers)
      (broadcast-condition emptyp)))
  t)

(define-monitor-method writer-enter ((rw read-write-lock))
  (with-slots (writerp readers emptyp) rw
    (loop :while (or writerp (positivep readers))
          :do (wait-on-condition emptyp))
    (setf writerp t))
  t)

(define-monitor-method writer-leave ((rw read-write-lock))
  (with-slots (writerp emptyp) rw
    (setf writerp nil)
    (broadcast-condition emptyp)
  t)
\end{verbatim}

Čtenáři čekají dokud je v místnosti písař a písaři čekají dokud jsou v
místnosti čtenáři nebo písař. Při výstupu z místnosti, poslední čtenář
signalizuje, že nikdo není v místnosti a v případě odchodu písaře jsme
použili k signalizaci metodu \texttt{broadcast-condition}, která
probudí všechna čekající vlákna. Tady je to potřeba, jelikož v
případě, že čeká více čtenářů, chceme probudit všechny a nemůžeme
vědět, kolik jich na podmínce čeká.

V tomto řešení mají stejně jako u prvního řešení se semafory přednost
čtenáři. Abychom dali přednost písařům, je potřeba uspat všechny
příchozí čtenáře, jakmile je před vchodem první písař a ten musí
počkat než budou pryč všichni čtenáři. Jednoduchou úpravou metody
\texttt{writer-enter} tohoto docílíme.

\begin{verbatim}
(define-monitor-method writer-enter ((rw read-write-lock))
  (with-slots (writerp readers emptyp) rw
    (loop :while writerp
          :do (wait-on-condition emptyp))
    (setf writerp t)
    (loop :while (positivep readers)
          :do (wait-on-condition emptyp)))
  t)
\end{verbatim}

Jakmile přijde první písař, nastaví slot \texttt{writerp} na
\texttt{t} a tím zajistí, že už žádní noví čtenáři neprojdou. Pak už
stačí akorát počkat, než všichni místnost opustí.

Zde by bylo ještě idiomatické přejmenovat podmíněnou proměnnou na
\texttt{nowriterp} a přidat ještě navíc \texttt{noreaderp}. Nicméně
řešení je funkční i takto.

\printbibliography

\end {document}
